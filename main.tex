\documentclass[14pt]{extarticle}
\usepackage{amsmath}
\usepackage{graphicx}
\usepackage{booktabs}
\usepackage[T2A]{fontenc}
\usepackage[utf8]{inputenc}
\usepackage[english,russian]{babel}
\usepackage{float}

\begin{document}
\pagestyle{empty}
\begin{center}
    Санкт-Петербургский Политехнический Университет Петра Великого

    \vspace{0.3cm}

    Физико-механический институт

    \vspace{0.3cm}

    Высшая школа прикладной математики и вычислительной физики

    \vspace{3cm}

    {\large\textbf{Отчет по лабораторным работам по №1-3 по математической статистике}}

    \vspace{4.5cm}

    Выполнил: \hspace{5.5cm}Андреев Даниил Витальевич

    Группа: \hspace{9.5cm}5030102/10101

    Преподаватель: \hspace{3.6cm}Баженов Александр Николаевич

    \vspace{4cm}

    Санкт-Петербург

    2024 год
\end{center}

\newpage

\section{Описательная статистика}
\subsection{Постановка задачи}
Для 5 распределений:

\begin{itemize}
    \item Нормальное распределение \(N(x, 0, 1)\)
    \item Распределение Коши \(C(x, 0, 1)\)
    \item Распределение Стьюдента \(t(x, 0, 3)\)
    \item Распределение Пуассона \(P(k, 10)\)
    \item Равномерное распределение \(U(x,-\sqrt{3}, \sqrt{3})\)
\end{itemize}\\
Сгенерировать выборки размером 10, 50, 1000 элементов. Построить на одном рисунке гистограмму и график плотности распределения.

\subsection{Теоретическая справка}

Плотности классических распределений:
\begin{itemize}
    \item Нормальное распределение \(N(x, 0,1)=\frac{1}{\sqrt{2\pi}}e^{-\frac{x^2}{2}}\) (Ф. 1.1)
    \item Распределение Коши \(C(x,0,1)=\frac{1}{\pi (x^2+1)}\) (Ф. 1.2)
    \item Распределение Стьюдента \(t(x,0,3)=\frac{\Gamma(2)(1+\frac{x^2}{3})^{-2}}{\sqrt{3\pi}\Gamma(\frac{3}{2})}\)
    (Ф. 1.3)
    \item Распределение Пуассона \(P(k, 10)=\frac{e^{-10}10^k}{k!}\) (Ф. 1.4)
    \item Равномерное распределение \(U(x, -\sqrt{3}, \sqrt{3})=\begin{cases}
    \frac{1}{2\sqrt{3}}&\text{, } |x|\leq\sqrt{3} \\
    0&\text{, } |x|>\sqrt{3}
\end{cases})\) (Ф. 1.5)
\end{itemize}\\\\
Графически гистограмма строится следующим образом: сначала множество значений, которое может принимать элемент выборки, разбивается на несколько интервалов. Чаще всего эти интервалы берут одинаковыми, но это не является строгим требованием. Эти интервалы откладываются на горизонтальной оси, затем над каждым рисуется прямоугольник. Если все интервалы были одинаковыми, то высота каждого прямоугольника пропорциональна числу элементов выборки, попадающих в соответствующий интервал. Если интервалы разные, то высота прямоугольника выбирается таким образом, чтобы его площадь была пропорциональна числу элементов выборки, которые попали в этот интервал.\\
Построение гистограмм используется для получения эмпирической оценки плотности распределения случайной величины. Они являются хорошим инструментом для исследования неизвестных распределений.

\subsection{Реализация}

Реализовывать гистограммы для этих распределений будем на языке Python3 с использованием пакетов numpy, matplotlib, scipy.
Ссылка на Github-репозиторий: https://github.com/Highlandet/MathStatisctics-Labs

\subsection{Результаты}

\begin{figure}[H]
    \centering
    \includegraphics[width=1.1\textwidth]{hauss.png}
    \caption{Нормальное распределение \(N(x, 0, 1)\) (Ф. 1.1)}
    \label{fig:enter-label}
\end{figure}

\begin{figure}[H]
    \centering
    \includegraphics[width=1.1\textwidth]{cauchy.png}
    \caption{Распределение Коши \(C(x, 0, 1)\) (Ф. 1.2)}
    \label{fig:enter-label}
\end{figure}

\begin{figure}[H]
    \centering
    \includegraphics[width=1.1\textwidth]{student.png}
    \caption{Распределение Стьюдента \(t(x, 0, 3)\) (Ф. 1.3)}
    \label{fig:enter-label}
\end{figure}

\begin{figure}[H]
    \centering
    \includegraphics[width=1.1\textwidth]{poisson.png}
    \caption{Распределение Пуассона \(P(k, 10)\) (Ф. 1.4)}
    \label{fig:enter-label}
\end{figure}

\begin{figure}[H]
    \centering
    \includegraphics[width=1.1\textwidth]{uniform.png}
    \caption{Равномерное распределение \(U (x, -\sqrt{3}, \sqrt{3})\) (Ф. 1.5)}
    \label{fig:enter-label}
\end{figure}

\subsection{Вывод}

Анализ результатов показывает, что чем больше выборка из распределения, тем больше соответствует гистограмма распределения графику плотности распределения графику плотности его распределения. Аналогично: чем больше выборка, тем более заметен характер распределения.

\section{Точечное оценивание характеристик положения и рассеяния}

\subsection{Постановка задачи}

Сгенерировать выборки размером 10, 100, 1000 элементов. Для каждой выборки вычислить следующие статистические характеристики положения данных: \(\overline{x}\), \(\text{med }x\), \(z_R\), \(z_Q\), \(z_{tr}\). Повторить такие вычисления 1000 раз для каждой выборки и найти среднее характеристик положения и их квадратов: \(E(z)=\overline{z}\), вычислить оценку дисперсии по формуле \(D(z)=\overline{z^2}-\overline{z}^2\). Представить полученные данные в виде таблиц

\subsection{Теоретическая справка}

Характеристики положения:
\begin{itemize}
    \item Выборочное среднее: \(\overline{x} = \frac{1}{n}\sum_{i=1}^n x_i\)
    \item Выборочная медиана: \(\text{med } x = \begin{cases}
    x_{(l+1)}&\text{, } n=2l+1 \\
    \frac{x_{(l)}+x_{(l+1)}}{2}&\text{, } n=2l\end{cases}\)
    \item Полусумма экстремальных выборочных элементов \(z_R=\frac{x_{(1)}+x{(n)}}{2}\)
    \item Полусумма квартилей: \(z_Q=\frac{z_{1/4}+z_{3/4}}{2}\)   \(\big(z_p=\begin{cases}
    x_{[np]+1}&\text{, } np\notin\mathbb{Z} \\
    x_{(np)}&\text{, } np\in\mathbb{Z}\end{cases}\big)\)
    \item Усеченное среднее: \(z_{tr}=\frac{1}{n-2r}\sum_{i=r+1}^{n-r} x_i, r\approx\frac{n}{4}\)
    \item Среднее характеристик положения: \(E(z)=\overline{z}\)
    \item Оценка дисперсии \(D=\overline{z^2}-\overline{z}^2\)
\end{itemize}

\subsection{Реализация}

Для вычисления статических характеристик положения данных будем использовать язык программирования Python3 и следующие библиотеки: numpy, scipy\\\\
Ссылка на Github-репозиторий: https://github.com/Highlandet/MathStatisctics-Labs

\subsection{Результаты и выводы}

\begin{table}[htbp]
    \centering
    \begin{tabular}{|c|c|c|c|c|c|}
        \toprule
        \textbf{Normal: n=10} & \(\overline{x}\) & med \(x\) & \(z_R\) & \(z_Q\) & \(z_{tr}\)\\
        \(E(z)\) & 0.02 & 0.04 & 0.03 & 0.00 & 0.03  \\
        \(D(z)\) & 0.10 & 0.47 & 0.54 & 0.12 & 0.44 \\
        \midrule
  	\textbf{Normal: n=10} & \(\overline{x}\) & med \(x\) & \(z_R\) & \(z_Q\) & \(z_{tr}\)\\
        \(E(z)\) & -0.00 & 0.01 & 0.00 & 0.02 & -0.05   \\
        \(D(z)\) & 0.01 & 0.44 & 0.50 & 0.12 & 3.23 \\
        \midrule
	\textbf{Normal: n=1000} & \(\overline{x}\) & med \(x\) & \(z_R\) & \(z_Q\) & \(z_{tr}\)\\
        \(E(z)\) & 0.00 & 0.04 & 0.00 & 0.00 & 0.31   \\
        \(D(z)\) &  0.00 & 0.49 & 0.48 & 0.12 & 31.42 \\
        \toprule
    \end{tabular}
    \caption{Распределение Коши \(C(x, 0, 1)\)}
    \label{tab:cauchy_t}
\end{table}

\begin{table}[htbp]
    \centering
    \begin{tabular}{|c|c|c|c|c|c|}
        \toprule
        \textbf{Cauchy: n=10} & \(\overline{x}\) & med \(x\) & \(z_R\) & \(z_Q\) & \(z_{tr}\)\\
        \(E(z)\) & 28.87 & 2.24 & -0.40 & 0.53 & 72.39  \\
        \(D(z)\) & 832980.95 & 12179.47 & 690.99 & 864.51 & 5206081.55 \\
        \midrule
  	\textbf{Cauchy: n=100} & \(\overline{x}\) & med \(x\) & \(z_R\) & \(z_Q\) & \(z_{tr}\)\\
        \(E(z)\) & 0.26 & -2.32 & 1.44 & 1.62 & 21.14    \\
        \(D(z)\) & 766.00 & 13521.18 & 3098.81 & 1119.64 & 377110.51 \\
        \midrule
	\textbf{Cauchy: n=1000} & \(\overline{x}\) & med \(x\) & \(z_R\) & \(z_Q\) & \(z_{tr}\)\\
        \(E(z)\) & -0.41 & 0.80 & 1.56 & 0.29 & -81.71    \\
        \(D(z)\) & 114.82 & 1523.30 & 2008.94 & 75.47 & 2323042.52 \\
        \toprule
    \end{tabular}
    \caption{Распределение Коши \(C(x, 0, 1)\)}
    \label{tab:cauchy_t}
\end{table}

\begin{table}[htbp]
    \centering
    \begin{tabular}{|c|c|c|c|c|c|}
        \toprule
        \textbf{Student: n=10} & \(\overline{x}\) & med \( x\) & \(z_R\) & \(z_Q\) & \(z_{tr}\)\\
        \(E(z)\) & 0.01 & 0.06 & 0.01 & -0.03 & 0.01  \\
        \(D(z)\) & 0.31 & 1.27 & 1.86 & 0.37 & 1.25 \\
        \midrule
  	\textbf{Student: n=100} & \(\overline{x}\) & med \( x\) & \(z_R\) & \(z_Q\) & \(z_{tr}\)\\
        \(E(z)\) & -0.00 & 0.03 & -0.01 & 0.02 & -0.07   \\
        \(D(z)\) & 0.03 & 1.65 & 1.71 & 0.43 & 11.04 \\
        \midrule
	\textbf{Student: n=1000} & \(\overline{x}\) & med \( x\) & \(z_R\) & \(z_Q\) & \(z_{tr}\)\\
        \(E(z)\) & 0.00 & 0.05 & 0.04 & -0.02 & 0.17  \\
        \(D(z)\) & 0.00 & 1.61 & 1.13 & 0.31 & 97.77  \\
        \toprule
    \end{tabular}
    \caption{Распределение Стьюдента \(t(x, 0, 3)\)}
    \label{tab:student_t}
\end{table}

\begin{table}[htbp]
    \centering
    \begin{tabular}{|c|c|c|c|c|c|}
        \toprule
        \textbf{Poisson: n=10} & \(\overline{x}\) & med \(x\) & \(z_R\) & \(z_Q\) & \(z_{tr}\)\\
        \(E(z)\) & 10.06 & 9.98 & 10.12 & 5.04 & 17.60  \\
        \(D(z)\) & 1.10 & 4.97 & 5.57 & 1.36 & 4.74 \\
        \midrule
  	\textbf{Poisson: n=100} & \(\overline{x}\) & med \(x\) & \(z_R\) & \(z_Q\) & \(z_{tr}\)\\
        \(E(z)\) & 10.01 & 10.02 & 9.96 & 5.04 & 127.65    \\
        \(D(z)\) & 0.10 & 5.15 & 5.36 & 1.21 & 33.42 \\
        \midrule
	\textbf{Poisson: n=1000} & \(\overline{x}\) & med \(x\) & \(z_R\) & \(z_Q\) & \(z_{tr}\)\\
        \(E(z)\) & 10.00 & 9.97 & 10.11 & 5.00 & 1252.60  \\
        \(D(z)\) & 0.01 & 5.13 & 5.32 & 1.22 & 314.38 \\
        \toprule
    \end{tabular}
    \caption{Распределение Пуассона \(P(10, k)\)}
    \label{tab:poisson_t}
\end{table}

\begin{table}[htbp]
    \centering
    \begin{tabular}{|c|c|c|c|c|c|}
        \toprule
        \textbf{Uniform: n=10} & \(\overline{x}\) & med \(x\) & \(z_R\) & \(z_Q\) & \(z_{tr}\)\\
        \(E(z)\) & 0.01 & 0.01 & 0.01 & 0.03 & 0.02   \\
        \(D(z)\) & 0.11 & 0.50 & 0.48 & 0.13 & 0.46 \\
        \midrule
  	\textbf{Uniform: n=100} & \(\overline{x}\) & med \(x\) & \(z_R\) & \(z_Q\) & \(z_{tr}\)\\
        \(E(z)\) & 0.00 & 0.02 & -0.04 & 0.00 & 0.05 \\
        \(D(z)\) & 0.01 & 0.48 & 0.52 & 0.13 & 3.16 \\
        \midrule
	\textbf{Uniform: n=1000} & \(\overline{x}\)& med \(x\) & \(z_R\) & \(z_Q\) & \(z_{tr}\)\\
        \(E(z)\) & -0.00 & -0.02 & -0.05 & -0.01 & -0.19  \\
        \(D(z)\) & 0.00 & 0.49 & 0.46 & 0.12 & 30.61 \\
        \toprule
    \end{tabular}
    \caption{Равномерное распределение \(U(x, -\sqrt{3}, \sqrt{3})\)}
    \label{tab:uniform_t}
\end{table}

Проанализировав полученные результаты, можно заметить, что для нормального распределения, распределения Стьюдента и равномерного распределения \(E(z)\) и \(D(z)\) для всех характеристик уменьшается с ростом выборки.\\
В таблице характеристик распределения Коши можно выделить аномальные значения, явно превышающие ожидаемые. Такой результат можно объяснить наличием различных выбросов, неопределенностью матожидания и бесконечностью дисперсии СВ, распределенной по данному закону.

\section{Бокс-плот Тьюки}

\subsection{Теоретическая справка}

Боксплот (англ. box plot) - график, использующийся в описательной статистике, компактно изображающий одномерное распределение вероятностей. Такой вид диаграммы в удобной форме показывает медиану, нижний и верхний квартили и выбросы. Границами ящика служат первый и третий квартили, линия в середине ящика - медиана. Концы усов - края статистически значимой выборки (без выбросов). Длину "усов" определяют разность первого квартиля и полутора межквартильных расстояний и сумма третьего квартиля и полутора межквартильных расстояний. Формула имеет вид:
\[X_1 = Q_1-\frac{3}{2}(Q_3-Q_1), X_2=Q_3+\frac{3}{2}(Q_3-Q1)\], \\
где \(X_1\) - нижняя граница уса, \(X_2\) - верхняя граница уса, \(Q_1\) - первый квартиль, \(Q_3\) - третий квартиль.

Данные, выходящие за границы усов (выбросы), отображаются на графике в виде маленьких кружков.

Выбросами считаются такие величины \(x\), что:
$\left[
      \begin{gathered}
        x < X_1^T; \\
        x > X_2^T, \\
      \end{gathered}
\right.$

\subsection{Постановка задачи}

Сгенерировать выборки размером 20 и 100 элементов. Построить для них боксплот Тьюки.

\subsection{Реализация}

Отрисовыват ь боксплоты будем на языке Python3 с использованием программных пакетов numpy и matplotlib.\\
Ссылка на Github-репозиторий: https://github.com/Highlandet/MathStatisctics-Labs/tree/main/L3-4

\subsection{Результаты}

\begin{figure}[H]
    \centering
    \includegraphics[width=1.1\textwidth]{boxplot_hauss.png}
    \caption{Нормальное распределение \(N(x, 0, 1)\) (Ф. 1.1)}
    \label{fig:enter-label}
\end{figure}

\begin{figure}[H]
    \centering
    \includegraphics[width=1.1\textwidth]{boxplot_cauchy.png}
    \caption{Распределение Коши \(C(x, 0, 1)\) (Ф. 1.2)}
    \label{fig:enter-label}
\end{figure}

\begin{figure}[H]
    \centering
    \includegraphics[width=1.1\textwidth]{boxplot_student.png}
    \caption{Распределение Стьюдента \(t(x, 0, 3)\) (Ф. 1.3)}
    \label{fig:enter-label}
\end{figure}

\begin{figure}[H]
    \centering
    \includegraphics[width=1.1\textwidth]{boxplot_poisson.png}
    \caption{Распределение Пуассона \(P(k, 10)\) (Ф. 1.4)}
    \label{fig:enter-label}
\end{figure}

\begin{figure}[H]
    \centering
    \includegraphics[width=1.1\textwidth]{boxplot_histogram.png}
    \caption{Равномерное распределение \(U (x, -\sqrt{3}, \sqrt{3})\) (Ф. 1.5)}
    \label{fig:enter-label}
\end{figure}

\subsection{Выводы}

Боксплот нормального распределения имеет симметричную форму ящика, почти по центру которого располагается медиана.\\
Для распределения Коши наблюдается смещение среднего значения от медианы и аномальные выбросы из-за отсутствия конечного матожидания и дисперсии.\\
При увеличении выборки ящик боксплота распределения Стьюдента смещается (это связано с ненулевым коэффициентом асимметрии при \(n\leq 3\)), и появляется больше выбросов.\\
Для распределения Пуассона боксплот Тьюки может более компактен и симмитричен при увеличении выборки, однако, как и в случае с другими распределениями, при небольших выборках или наличии выбросов наблюдаются некоторые асимметрии (смещен ящик боксплота).\\
Для равномерного распределения с увеличением выборки стремятся поделить в соответствующем процентном соотношении всю длину боксплота - это связано с тем, что равномерное распределение не имеет выбросов.

\end{document}
